Растущая популярность\cite{XRgrows} технологий компьютерного зрения и
расширенной реальности влечёт за собой потребность в способах компактного
хранения и передачи трёхмерных (далее 3Д) данных.

Обычным форматом представления 3Д-данных являются полигональные сетки.
Полигональная сетка представляет собой совокупность вершин, рёбер и граней,
определяющих форму трёхмерного объекта. Подобным образом представляются
трёхмерные объекты и сцены в видеоиграх, 3Д-анимации и программах для
компьютерного моделирования.

Однако, существуют приложения, в которых необходим иной способ представления
данных - облака точек. Облако точек представляет собой конечное множество точек
в трёхмерном пространстве, а также атрибуты данных точек, такие как цвет или
нормаль к поверхности, на которой лежит точка. Облака точек являются выходным
форматом 3Д-сканеров - устройств, позволяющих считывать форму физического
объекта и его внешние характеристики, такие как цвет или отражающая способность
поверхности\cite[10]{SurfaceReconstruction}. Данный формат представления
3Д-данных используется в приложениях, использующих 3Д-сканирование для получения
информации об окружающей среде, к подобным приложениям относятся компьютерное
зрение и системы расширенной реальности.

% Полигональные сетки аппроксимируют непрерывную поверхность исходного объекта. В
% свою очередь, облака точек могут сохранять мельчайшие подробности структуры
% поверхности объекта вплоть до миллиметра\cite[33]{SurfaceReconstruction}.
% Большая точность позволяет использовать облака точек для машинной обработки в
% системах компьютерного зрения, взаимодействующих с реальным миром.


% Примером подобных систем являются беспилотные автомобили, использующие лидары -
% локаторы, испускающие световые волны оптического диапазона с дальнейшей
% регистрацией отраженных импульсов\cite[7]{PointCloudAnalysis}. Другим примером
% являются системы расширенной реальности, в данном случае облака точек
% используются для совмещения позиций виртуальных объектов и физических объектов,
% находящихся рядом с человеком\cite[15]{PointCloudAnalysis}.


%  Задача получения 3Д-модели объекта в виде полигональной сетки по имеющемуся
% облаку точек, считанных с его поверхности, решается методами реконструкции
% поверхности\cite{SurfaceReconstruction}. Для обратного преобразования достаточно
% удалить связи между вершинами в полигональной сетке.


% Для точного описания поверхности, облака точек должны быть достаточно плотными.
% Точки в облаке представляют собой дискретные образцы непрерывной поверхности, а
% полигональные сетки аппроксимируют данную поверхность
% полигонами\cite[4]{PointCloudAnalysis}. Всё это влечёт за собой большой размер
% облаков точек (может тут тоже ссылка?), в связи с чем возникает задача
% разработки способов компактного хранения подобных объектов.

Алгоритмы сжатия облаков точек решают задачу компактного хранения и передачи
облаков точек. Популярные кодеки были реализованы в рамках проектов
Draco\cite{Draco} и PCL\cite{Rusu_ICRA2011_PCL}. В настоящее время различными
авторами предлагаются новые методы сжатия геометрической структуры облаков
точек\cite{PCGCv2}\cite{GeoCNNv2} и их
атрибутов\cite{Shao2017}\cite{Chen2020}\cite{Sun2023}. Работа по стандартизации
PCC-кодеков была начата MPEG в 2017 году\cite{CallForProposalV2}, также данной
группой был предложен собственный кодек и разработана тестовая модель на его
основе - TMC13\cite{TMC13}. Алгоритм кодирования данных в этих решениях можно
разделить на этапы реконструкции изначальной геометрической структуры объекта и
кодирования атрибутов облака в соответствии с полученной структурой.

Большое количество постоянно появляющихся подходов к сжатию облаков точек делает
актуальной задачу разработки программы для оценки работы PCC-кодеков. Подобная
программа может быть использована исследователями для подсчёта метрик
разрабатываемых ими кодеков.

\textbf{Цель работы} - разработка подхода к сравнению алгоритмов сжатия
атрибутов облаков точек. В рамках данной работы необходимо решить следующие
задачи:

\begin{itemize}
    \item Проанализировать системы оценки качества сжатия облаков точек
    \item Изучить релевантные метрики, отображающие эффективность и качество
    сжатия атрибутов облаков точек
    \item Разработать программу подсчёта метрик
    \item Получить метрики для отобранных PCC-кодеков
    \item Проанализировать результаты работы
\end{itemize}

% \textbf{Алгоритмы сжатия облаков точек}.  Алгоритмы сжатия облаков точек
% (PCC-алгоритмы - Point Cloud Compression алгоритмы) можно классифицировать по
% следующим признакам:

% \begin{itemize}
%     \item Signal Processing / ML-based\cite{Wu2020}
%     \item С потерями / без потерь геометрии\cite{CallForProposalV2}
%     \item С потерями / без потерь атрибутов\cite{CallForProposalV2}
% \end{itemize}
