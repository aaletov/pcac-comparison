В данной работе был проведён анализ существующих систем оценки методов сжатия
облаков точек и их атрибутов, а также определён набор метрик для оценки качества
сжатия атрибутов облаков точек. На основе проведённого анализа разработана
программа для оценки качества облака точек при наличии оригинального облака
точек. Произведён качественный сравнительный анализ кодеков Draco и TMC13.
Полученные данные позволяют судить о качестве сжатия облаков точек данными
кодеками при различной степени сжатия. Разработанное решение упростит оценку
методов сжатия атрибутов облаков точек и может быть полезным исследователям,
ведущим разработки в данной области, а также может быть использовано при
разработке проприетарных кодеков. В рамках дальнейшей работы в программу могут
быть добавлены другие качественные метрики, дающие более подробную оценку
качества сжатия атрибутов облаков точек.
