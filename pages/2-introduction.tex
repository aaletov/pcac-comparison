\textbf{Облака точек}. Растущая популярность (?ссылка?) технологий компьютерного
зрения (CV - Computer Vision) и расширенной реальности (XR - eXtended Reality)
влечёт за собой потребность в способах компактного хранения и передачи
трёхмерных (далее 3Д) данных. 3Д-данные представляются в виде полигональных
сеток - совокупности вершин, рёбер и граней, определяющих форму трёхмерного
объекта, или облаков точек, отличающихся от последних отсутствием связей между
вершинами.

Цикл жизни любых данных состоит из этапов создания, преобразования и передачи.
3Д-данные могут быть получены с помощью программ для компьютерного моделирования
или посредством 3Д-сканирования - считывания формы физического объекта и его
внешних характеристик, таких как цвет или отражающая способность поверхности.

В общем случае, результатом процесса 3Д-сканирования является конечное множество
точек в трёхмерном пространстве\cite[10]{SurfaceReconstruction}. Большинство
технологий позволяет считывать точки лишь с поверхности поверхности объекта,
исключением является технология промышленной компьютерной томографии,
позволяющая считывать точки не только с поверхности объекта, но и из его
внутреннего объема.

Облака точек не являются описаниями поверхностей, в отличие от полигональных
сеток. Полигональные сетки содержат грани и рёбра, аппроксимирующие непрерывную
поверхность исходного объекта. Реальные объекты воспринимаются человеком как
непрерывная поверхность, вследствие чего полигональные сетки лучше подходят (кто
сказал??) с точки зрения восприятия объекта человеком или использования объекта
в программах для компьютерного моделирования. Задача получения 3Д-модели объекта
в виде полигональной сетки по имеющемуся облаку точек, считанных с его
поверхности, решается методами реконструкции
поверхности\cite{SurfaceReconstruction}. (тут можно сказать что плотные облака
тоже подходят кожаным (ссылки есть на вики в статье со сканом здания в алеппо))

Реконструкция поверхности не всегда является необходимой, в некоторых
приложениях информация о поверхности, содержащаяся в облаке точек, может быть
обработана машинными методами напрямую. Облака точек используются в системах
компьютерного зрения на основе лидаров - локаторов, использующих технологию
испускания световых волн оптического диапазона с дальнейшей регистрацией
отраженных импульсов, к подобным системам относятся беспилотные
автомобили\cite[7]{PointCloudAnalysis}. Другим примером систем, использующих
облака точек, являются системы (тавтология) расширенной реальности, в данном
случае облака точек используются для совмещения позиций виртуальных объектов и
физических объектов, находящихся рядом с человеком\cite[15]{PointCloudAnalysis}.

Таким образом, облака точек являются компромиссом между различными форматами
хранения 3Д-данных по простоте сканирования, реалистичности рендеринга, удобству
манипуляции и обработки (кто сказал?).

\textbf{Алгоритмы сжатия облаков точек}. 3Д-данные занимают гораздо больше места
по сравнению с более распространенными двумерными представлениями. Цифры, цифры,
цифры. Принципиально бОльший размер данных влечёт за собой потребность в
компактном хранении. Алгоритмы сжатия облаков точек (PCC-алгоритмы - Point Cloud
Compression алгоритмы) можно классифицировать по следующим признакам:

\begin{itemize}
    \item Signal Processing / ML-based (?)
    \item С потерями / без потерь (?) геометрии
    \item С потерями / без потерь (?) атрибутов
\end{itemize}

Были разработаны различные кодеки для сжатия облаков точек, среди которых Draco,
PCL и TMC13. (выбор наверное не во введении надо обосновывать?). Алгоритм
кодирования данных в этих решениях можно разделить на этапы воссоздания
изначальной геометрии объекта и кодирования атрибутов облака в соответствии с
полученной структурой (?).

\textbf{Задача сжатия атрибутов облаков точек}.

Текст. Текст. Текст.

\textbf{Цель работы} - разработка подхода к сравнению алгоритмов сжатия
атрибутов облаков точек. В рамках данной работы необходимо решить следующие
задачи:

\begin{itemize}
    \item Проанализировать существующие PCC-кодеки
    \item Изучить релевантные метрики, отображающие эффективность (??) /
    качество (??) сжатия атрибутов облаков точек
    \item Разработать программу подсчёта метрик
    \item Получить метрики для отобранных PCC-кодеков
    \item Проанализировать результаты работы
\end{itemize}


