\textbf{Облака точек}. Растущая популярность (?ссылка?) технологий компьютерного
зрения (CV - Computer Vision) и расширенной реальности (XR - eXtended Reality)
влечёт за собой потребность в способах компактного хранения и передачи
трёхмерных (далее 3Д) данных. 3Д-данные представляются в виде полигональных
сеток - совокупности вершин, рёбер и граней, определяющих форму трёхмерного
объекта, или облаков точек, отличающихся от последних отсутствием связей между
вершинами.

Цикл жизни любых данных состоит из этапов создания, преобразования и передачи.
3Д-данные могут быть получены с помощью программ компьютерного моделирования или
посредством считывания физического объекта специального устройства - 3Д-сканера.
Существующие устройства 3Д-сканирования позволяют считывать лишь отдельные
вершины физического объекта (кто сказал?), установления связей между вершинами
возможно лишь в ходе обработки полученных данных специализированными алгоритмами
(кто сказал?).

Наиболее точным (кто сказал?) методом 3Д-сканирования на данный момент является
использование лидаров - лазерный локатор[1], использующий технологию испускания
лазером волн оптического диапазона с дальнейшей регистрацией лазерных импульсов,
которые были рассеяны объектами: лазерную[2][3] (или оптико-электронную[4])
локацию. Результатом работы лидара является множество отдельных вершин с
координатами x, y, z - облако точек.

Таким образом, облака точек являются компромиссом между различными форматами
хранения 3Д-данных по простоте сканирования, реалистичности рендеринга, удобству
манипуляции и обработки (кто сказал?).

\textbf{Алгоритмы сжатия облаков точек}. 3Д-данные занимают гораздо больше места
по сравнению с более распространенными двумерными представлениями. Цифры, цифры,
цифры. Принципиально бОльший размер данных влечёт за собой потребность в
компактном хранении. Алгоритмы сжатия облаков точек (PCC-алгоритмы - Point Cloud
Compression алгоритмы) можно классифицировать по следующим признакам:

\begin{itemize}
    \item Signal Processing / ML-based (?)
    \item С потерями / без потерь (?) геометрии
    \item С потерями / без потерь (?) атрибутов
\end{itemize}

Были разработаны различные кодеки для сжатия облаков точек, среди которых Draco,
PCL и TMC13. (выбор наверное не во введении надо обосновывать?). Алгоритм
кодирования данных в этих решениях можно разделить на этапы воссоздания
изначальной геометрии объекта и кодирования атрибутов облака в соответствии с
полученной структурой (?).

\textbf{Задача сжатия атрибутов облаков точек}.

Текст. Текст. Текст.

\textbf{Цель работы} - разработка подхода к сравнению алгоритмов сжатия
атрибутов облаков точек. В рамках данной работы необходимо решить следующие
задачи:

\begin{itemize}
    \item Проанализировать существующие PCC-кодеки
    \item Изучить релевантные метрики, отображающие эффективность (??) /
    качество (??) сжатия атрибутов облаков точек
    \item Разработать программу подсчёта метрик
    \item Получить метрики для отобранных PCC-кодеков
    \item Проанализировать результаты работы
\end{itemize}


