\begin{center}
  {САНКТ-ПЕТЕРБУРГСКИЙ ПОЛИТЕХНИЧЕСКИЙ УНИВЕРСИТЕТ\\ ПЕТРА ВЕЛИКОГО} \\
  {Институт компьютерных наук и кибербезопасности} \\
  {Высшая школа программной инженерии} \\

  {
  \begin{flushright}
    УТВЕРЖДАЮ\\
    Директор ВШПИ\\
    \underline{\hspace{2.2cm}} П.Д. Дробинцев\\
    <<\underline{\hspace*{0.05\textheight}}>> \underline{\hspace*{0.1\textheight}} \the\year{}~г. \\[1.8cm]
  \end{flushright}
  }

  \textbf{ЗАДАНИЕ} \\
  \textbf{на выполнение выпускной квалификационной работы}\\
  студенту \AuthorFull, группа \Group \\[0.5cm]
  \begin{enumerate}[label=\arabic*.]
    \item Тема работы: \Theme
    \item Срок сдачи студентом законченной работы: \TaskDeadline
    \item Исходные данные по работе:
      \begin{itemize}
        \item Документация на язык программирования Python;
        \item Документация к библиотеке Open3D;
        \item Документация к библиотеке NumPy;
      \end{itemize}
    \item Содержание работы (перечень подлежащих разработке вопросов):
      \begin{itemize}
        \item Обоснование актуальности работы
        \item Обзор существующих решений
        \item Составление требований к разрабатываемому решению
        \item Описание реализации
        \item Анализ результатов
      \end{itemize}
    \item Перечень графического материала (с указанием обязательных чертежей):
    \item Консультанты по работе: старший преподаватель ВШПИ ИКНК Шемякин И.А.
    \item Дата выдачи задания: \TaskCreatedCiframi
  \end{enumerate}
\end{center}
Руководитель ВКР \hspace{3.5cm} \underline{\hspace{5cm}} \hfill \Supervisor\\[0.5cm]
Задание принял к исполнению: \TaskCreatedCiframi\\[0.5cm]
Студент \hspace{5.8cm} \underline{\hspace{5cm}} \hfill \Author\\
