Бакалаврская работа посвящена количественному сравнению популярных алгоритмов
сжатия атрибутов облаков точек. Дан обзор существующих алгоритмов и
распространённых кодеков с открытым исходным кодом. В результате сравнительного
анализа алгоритмов получены данные об эффективности (!!!) отобранных алгоритмов
сжатия атрибутов облаков точек при различных данных и входных параметрах.

В рамках работы была разработана реализация отобранных алгоритмов с
использованием языка C++ в качестве расширения к тестовой модели TMC13 -
реализации алгоритма сжатия облаков точек G-PCC, предложенной группой MPEG.

Выбран самый эффективный алгоритм, разработана программа для навигации
роботов-бобров на местности Оренбургской области.
