\documentclass[a4paper,12pt]{extreport}
\usepackage[left=20mm,right=10mm,top=20mm,bottom=20mm]{geometry}
% \parindent 1.25cm
% \linespread{1.5}

\usepackage{cmap}
\usepackage{lipsum}

\usepackage[utf8]{inputenc}
\usepackage[T2A]{fontenc}
\usepackage[english,russian]{babel}
\usepackage{fontspec}
\setmainfont{Times New Roman}
\setmonofont{Consolas}

\usepackage{color}
\usepackage{listings}

\lstset{ %
    language=python,                % выбор языка для подсветки (здесь это С)
    basicstyle=\small\ttfamily,   % размер и начертание шрифта для подсветки кода
    numbers=left,                 % где поставить нумерацию строк (слева\справа)
    numberstyle=\tiny,            % размер шрифта для номеров строк
    stepnumber=1,                 % размер шага между двумя номерами строк
    firstnumber=1,
    numberfirstline=true
    numbersep=5pt,                % как далеко отстоят номера строк от подсвечиваемого кода
    backgroundcolor=\color{white},% цвет фона подсветки - используем \usepackage{color}
    showspaces=false,             % показывать или нет пробелы специальными отступами
    showstringspaces=false,       % показывать или нет пробелы в строках
    showtabs=false,               % показывать или нет табуляцию в строках
    frame=single,                 % рисовать рамку вокруг кода
    tabsize=2,                    % размер табуляции по умолчанию равен 2 пробелам
    captionpos=t,                 % позиция заголовка вверху [t] или внизу [b]
    breaklines=true,              % автоматически переносить строки (да\нет)
    breakatwhitespace=false      % переносить строки только если есть пробел
}

\usepackage{graphicx}
\usepackage{float}
\usepackage{subcaption}
\usepackage{xltabular, tabularx}

\usepackage[raggedright]{titlesec} % заголовки

\newcounter{slide}
\setcounter{slide}{1}

% \newcommand{\customparagraph}{(\arabic{slide}) \theparagraph}

\titleformat{\paragraph}[block]{\bfseries}{\theparagraph}{{|}}{(Слайд \arabic{slide}) }[\addtocounter{slide}{1}]
\titlespacing{\paragraph}{0pt}{10pt}{0pt}

\begin{document}

\paragraph{Титульник}

Добрый день, уважаемые члены комиссии, меня зовут Поздняков Артемий Анатольевич
и я хочу представить выпускную квалификационную работу бакалавра на тему
''Сравнение методов сжатия атрибутов облаков точек''. Научным руководителем
данной работы является Федоров Станислав Алексеевич.

\paragraph{Актуальность}

Растущая популярность технологий компьютерного зрения и расширенной реальности
влечёт за собой потребность в способах компактного хранения и передачи облаков
точек. В настоящее время появляется большое количество кодеков, предназначенных
для сжатия облаков точек и их атрибутов (PCC-кодеков), что делает актуальной
задачей разработку программы для оценки работы PCC-кодеков. Подобная программа
может быть использована исследователями для подсчёта метрик разрабатываемых ими
кодеков.

\paragraph{Цели и задачи}

\textbf{Цель работы} - разработка подхода к сравнению методов сжатия атрибутов
облаков точек. В рамках данной работы необходимо решить следующие задачи:

\begin{itemize}
    \item Проанализировать системы оценки качества сжатия облаков точек
    \item Изучить релевантные метрики, отображающие эффективность и качество
    сжатия атрибутов облаков точек
    \item Разработать программу подсчёта метрик
    \item Получить метрики для отобранных PCC-кодеков
    \item Проанализировать результаты работы
\end{itemize}

\paragraph{Оценка качества сжатия}

Сравнение кодеков будет осуществляться следующим образом: возьмем некоторое
контрольное облако точек, осуществим его компрессию и декомпрессию с помощью
некоторого кодека $C$, полученное в результате декомпрессии облако точек назовём
реконструированным облаком точек. Показателем качества сжатия данного облака
точек, то есть качественной оценкой работы кодека, будут являться значения
отобранных метрик для пары оригинальное и реконструированное облако.

\paragraph{Сравнение альтернативных решений}

Среди существующих решений, предназначенных для данной задачи, можно отметить
системы оценки mpeg\_pcc\_dmetric от MPEG и geo\_dist от авторов кодека
GeoCNNv1.

На слайде приведены характеристики данных систем оценки. Здесь, под полнотой
метрик подразумевается возможность подсчёта среднеквадратичной ошибки, отношения
пикового сигнала к шуму, а также значения данных метрик, проецированные вдоль
нормалей точек (нормаль точки - нормаль к плоскости, на которой лежит точка).
Что касается открытости исходного кода, решение от MPEG предоставляется лишь
исследователям по специальному запросу, а программа geo\_dist опубликована на
Github, но не обладает лицензией, что формально не даёт возможности данный код
использовать.

\paragraph{Требования}

\begin{itemize}
    \item Возможность вычисления стандартных метрик искажения геом. структуры (MSE и PSNR,
    метрика Хаусдорфа) геометрической структуры;
    \item Возможность вычисления проецированных значений отклонения;
    \item Возможность вычисления искажения цветов в цветовых схемах RGB и Y'CbCr;
    \item Использование архитектуры, допускающей дальнейшее расширение
    приложения;
    \item Наличие тестов;
    \item Использование лицензии MIT;
\end{itemize}

\paragraph{Архитектура разрабатываемого решения}

Для реализации данного проекта использовался язык Python. Диаграмма классов
разработанного приложения приведена на слайде.

\paragraph{Алгоритм внедрения зависимостей}

На слайде приведен алгоритм внедрения зависимостей, позволяющий избежать
повторного вычисления уже вычисленных метрик. Данный алгоритм реализован в
методе \texttt{recursive\_calculate} класса MetricCalculator.

\paragraph{Консольное приложение (help)}

Разработанное решение представляет собой консольное приложение. Help-сообщение
программы и входные параметры приведены на слайде. По умолчанию программа
выводит MSE и PSNR для координат точек, а также для цветов в цветовой схеме RGB,
клиент дополнительно может указать программе вычислить метрику Хаусдорфа,
значения метрик, проецированные вдоль нормалей (для MSE, PSNR координат и
метрики Хаусдорфа), а также указать цветовую схему, в которой должно вычисляться
искажение цветов. Дополнительно поддерживается вывод в формате CSV, что может
быть использовано при машинной обработке результатов.

\paragraph{Консольное приложение (вывод)}

Пример работы программы приведен на слайде, здесь отображены значения MSE и PSNR
для координат и цветов, а также минимальное и минимальное расстояние между
парами точек в оригинальном и реконструированном облаке.

\paragraph{Детали реализации}

На слайде приведена информация об используемых библиотеках, тестировании и
CI/CD.

\paragraph{Введение про PCCArena}

Разработанное решение было внедрено в платформу PCCArena. PCCArena представляет
собой систему бенчмаркинга PCC-кодеков. Данная система использует
mpeg\_pcc\_dmetric для вычисления математически-обоснованные метрик (MSE, PSNR,
и т.д.).

\paragraph{Архитектура PCCArena}

Архитектура PCCArena приведена на слайде. Для внедрения разработанного решения в
данную систему были внесены изменения в класс PointBasedMetrics.

\paragraph{Описание проведенных экспериментов}

С помощью модифицированной системы PCCArena была произведена оценка кодеков
TMC13 и Draco, использовался датасет ShapeNet. Для каждой метрики строится
зависимость от битрейта - количества бит, затраченных на кодирование одной
точки.

\paragraph{Результаты для расстояния Чамфера}

На данном слайде приведена зависимость расстояния Чамфера от битрейта, а также
зависимость соответствующего значения PSNR. Расстояние Чамфера считается так же
как MSE, это просто другое название. За пиковое значение сигнала берется
максимальное расстояние между всеми парами точек в оригинальном и
реконструированном облаке.

\paragraph{Результаты для метрики Хаусдорфа и нормалей}

На данном слайде приведена зависимость метрики Хаусдорфа от битрейта, а также
зависимость проецированной метрики Хаусдорфа.

\paragraph{Результаты для цветов}

На данном слайде приведена зависимость Y'-PSNR от битрейта. Y'-PSNR - это PSNR
для компоненты Y' цветовой схемы Y'CbCr. По приведенным графикам видно, что
Draco лишь иногда показывает меньшую ошибку при большом битрейте. Для TMC13
точки расположены более кучно вдоль одной кривой, что показывает большую
стабильность данного кодека. Можно сказать, что в среднем TMC13 всегда
показывает меньшую ошибку, чем Draco.

\paragraph{Выводы и дальнейшие шаги}

В данной работе был проведён анализ существующих систем оценки методов сжатия
облаков точек и их атрибутов. Разработана программа для оценки качества облака
точек при наличии оригинального облака точек. Произведён сравнительный анализ
кодеков Draco и TMC13. Полученные данные позволяют судить о качестве сжатия
облаков точек данными кодеками при различной степени сжатия. Разработанное
решение упростит оценку методов сжатия атрибутов облаков точек и может быть
полезным исследователям, ведущим разработки в данной области.

В рамках дальнейшей работы в программу могут быть добавлены метрики, учитывающие
более высокоуровневые признаки облаков точек и дающие более подробную оценку
качества их сжатия.

\subsection*{Extra}

\noindent Метрика - мера, значение, полученное в результате измерения.

\noindent Метрика - ф-я, удовл. аксиомам тождества, симметричности и нер-ву
треугольника.

\noindent SISIM: statistical information similarity-based point cloud quality
assessment

\end{document}